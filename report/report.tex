%%% Template originaly created by Karol Kozioł (mail@karol-koziol.net) and modified for ShareLaTeX use

\documentclass[a4paper,11pt]{ctexart}

\CTEXsetup[format={\Large\bfseries}]{section}

\usepackage[T1]{fontenc}
\usepackage[utf8]{inputenc}
\usepackage{graphicx}
\usepackage{xcolor}

\renewcommand\familydefault{\sfdefault}
\usepackage{tgheros}

\usepackage{amsmath,amssymb,amsthm,textcomp}
\usepackage{enumerate}
\usepackage{multicol}
\usepackage{tikz}

\usepackage{geometry}
\geometry{left=25mm,right=25mm,%
bindingoffset=0mm, top=20mm,bottom=20mm}

\usepackage{hyperref}
\hypersetup{hypertex=true,
            colorlinks=true,
            linkcolor=red,
            anchorcolor=red,
            citecolor=red}

\linespread{1.3}

\newcommand{\linia}{\rule{\linewidth}{0.5pt}}

% custom theorems if needed
\newtheoremstyle{mytheor}
    {1ex}{1ex}{\normalfont}{0pt}{\scshape}{.}{1ex}
    {{\thmname{#1 }}{\thmnumber{#2}}{\thmnote{ (#3)}}}

\theoremstyle{mytheor}
\newtheorem{defi}{Definition}

% my own titles
\makeatletter
\renewcommand{\maketitle}{
\begin{center}
\vspace{2ex}
{\huge \textsc{\@title}}
\vspace{1ex}
\\
\linia\\
\@author \hfill \@date
\vspace{4ex}
\end{center}
}
\makeatother
%%%

\usepackage{listings}
\lstset{
    language=C,
    basicstyle=\ttfamily\small,
    aboveskip={1.0\baselineskip},
    belowskip={1.0\baselineskip},
    columns=fixed,
    extendedchars=true,
    breaklines=true,
    tabsize=4,
    prebreak=\raisebox{0ex}[0ex][0ex]{\ensuremath{\hookleftarrow}},
    frame=lines,
    showtabs=false,
    showspaces=false,
    showstringspaces=false,
    keywordstyle=\color[rgb]{0.627,0.126,0.941},
    commentstyle=\color[rgb]{0.133,0.545,0.133},
    stringstyle=\color[rgb]{01,0,0},
    numbers=left,
    numberstyle=\small,
    stepnumber=1,
    numbersep=10pt,
    captionpos=t,
    escapeinside={\%*}{*)}
}

% custom footers and headers
\usepackage{fancyhdr}
\pagestyle{fancy}
\lhead{}
\chead{}
\rhead{}
\lfoot{形式语言与自动机实验报告}
\cfoot{}
\rfoot{\thepage}
\renewcommand{\headrulewidth}{0pt}
\renewcommand{\footrulewidth}{0pt}
%

%%%----------%%%----------%%%----------%%%----------%%%

\begin{document}

\title{形式语言与自动机实验报告}

\author{181220049 宋磊\ 人工智能学院}

\date{2020年12月22日}

\maketitle

\section{实验描述}

实验内容为实现多磁带确定性图灵机,能够读取和解析符合特定语法的图灵机
程序,并使用解析得到的图灵机程序对输入串进行判定。

我完成了所有内容,包括编写和解析图灵机文件,对图灵机程序的合法性判断,输入
的合法性判断,打印信息以及输出判定结果等。

\section{分析与设计思路}

具体代码见$tm.h$以及$tm.cpp$。

\subsection{数据结构的选择}

具体见$tm.h$代码。

图灵机类$Turing$使用$set<string>$存储状态集合$Q$、输入符号集合$S$、
磁带符号集合$G$以及终止状态集合$F$。使用$map$数据结构表示
转移函数,$map$的键为当前状态、当前磁带(将所有磁头下的符号连接
组成字符串作为磁带状态)状态的结构体,值为下一状态、
写入磁带的符号以及移动方向组成的结构体。使用$string$表示起始状态和空字符。

对单条磁带,使用结构体$tape\_t$记录了磁头
位置,磁带第一个非空字符的位置以及最后一个非空字符的位置,并使用$map$
存储所有的磁带上的状态(使用$map$主要是考虑到磁带的索引会出现负数
下标并且磁带长度无限,使用数组可能会遇到问题)。

\subsection{算法设计思路}

具体见$tm.cpp$代码。

首先,读取图灵机程序并进行解析和检查工作(具体见$Turing$类中$build\_tm$方法)。
解析过程主要是按行读取,
在删除分号之后的字符后,判断是否剩下非空格字符,如果有,则判断
该行的类型(可能为状态集合、输入集合、转移函数等),判断后更新图灵机存储
的内容。
检查工作主要是对转移函数的合法性的检查,包括,转移函数的当前状态、
下一状态是否属于$Q$,磁带状态长度是否等于磁带数$N$,磁带状态中的符号是否
属于$G$,是否存在同一个$key$映射到不同值的情况(非确定性转移)。

之后的模拟过程中(具体见$Turing$类中$simulate$方法),将当前状态以及磁带状态作为键,查询值,如果不存在,则
跳出模拟过程,结束模拟。如果存在,则转移,更新所有状态,其中比较重要的是
对磁带状态的更新,在移动磁头后,我会根据当前的磁头坐标和磁带记录的字符串起始
结束位置范($tape\_t$结构体中的$range$变量)进行比较,如果磁头超出范围,则增大范围,
如果小于范围,则判断两端是否存在空字符,如果存在则缩小范围直到不存在空字符。

\section{遇到的问题及解决方案}

遇到的问题主要是在编写图灵机程序时,容易漏写一些转移函数,使得在一半过程中
就退出。解决方法使用了$--verbose$参数可视化每一步转移过程,根据退出时的状态
增加新的转移函数,不断重复。在这个过程中也想起了jyy老师经常说编程时基础设施
($--verbose$可视化图灵机运作过程)的重要性,能够节约很多调试的时间。

\section{总结感想}

在完成实验的过程中,我对图灵机的运作方式的理解进一步加深,并且复习了很多图灵机的知识点。

这是一门比较硬核同时也十分重要的课程,学习过程中我了解到了FA、PDA、TM以及它们对应的
语言,以及很多计算机理论方面的知识点,希望在之后在面临实际问题时,也能想到用
自动机的方式进行建模和分析,用来帮助之后的研究工作。

最后也十分感谢老师和助教的辛苦工作。

\end{document}